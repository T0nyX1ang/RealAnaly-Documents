\documentclass[bwprint, withoutpreface]{cumcmthesis}

\usepackage{extarrows}
\usepackage{authblk}
\usepackage{tikz}
\usepackage{xcolor}
\usetikzlibrary{intersections, calc, bending, decorations.markings, arrows, shapes, positioning, decorations.pathreplacing, shadows, arrows.meta}
\newcommand*{\st}{\mathop{}\!\mathrm{s.t.}\!\mathop{}}
\newcommand*{\aev}{\mathop{}\!\mathrm{a.e.}\!\mathop{}}
\newcommand*{\dif}{\mathop{}\!\mathrm{d}}
\newcommand*{\norm}[1]{\| #1 \|}
\newcommand*{\nnorm}[2]{\| #1 \|_{#2}}
\newcommand*{\sgn}[1]{\mathrm{sgn}\!\mathop{}#1}

\title{泛函第一二章总结}

\begin{document}
\maketitle
\noindent Author: Tony Xiang

\noindent Full Document can be acquired here: 

\noindent https://github.com/T0nyX1ang/RealAnaly-Documents/blob/master/Chapter\%2056/Chapter56.pdf

\noindent Full Source code can be downloaded here:

\noindent https://github.com/T0nyX1ang/RealAnaly-Documents/blob/master/Chapter\%2056/Chapter56.tex

\section{距离空间的基本概念}
标量域:$\mathbb{K}$,既可以表示实数域,也可以表示复数域.

距离:$X \neq \emptyset, \forall x, y \in X, \exists d(x, y) \in \mathbb{R}^n, \st:$
\begin{itemize}[itemindent=2em]
	\item (正定性)$d(x, y) \geqslant 0, d(x, y) = 0 \iff x = y$;
	\item (对称性)$d(x, y) = d(y, x)$;
	\item (三角不等式)$d(x, y) \leqslant d(x, z) + d(z, y)$.
\end{itemize}

$(X, d)$称为距离空间.

距离空间的例子:
\begin{itemize}[itemindent=2em]
	\item 欧式空间$\mathbb{K}^n$.可以定义很多距离.
	\begin{align*}
		d_1(x, y) & = (\sum_{i = 1}^{n}{|x_i - y_i|^2})^{\frac{1}{2}} \\
		d_2(x, y) & = \sum_{i = 1}^{n}{|x_i - y_i|} \\
		d_3(x, y) & = \max_{1 \leqslant i \leqslant n}{|x_i - y_i|}.
	\end{align*}
	\item \textbf{连续函数空间}$C[a, b]$.设$C[a, b]$是$[a, b]$上连续函数全体.$\forall x = x(t), y = y(t)$,\[d(x, y) = \max_{a \leqslant t \leqslant b}{|x(t) - y(t)|}\] 
	\item 数列空间$s$.设$s$是实或复数列的全体.$\forall x = (x_i), y = (y_i)$,\[d(x, y) = \sum_{i = 1}^{\infty}{\frac{1}{2^i} \frac{|x_i - y_i|}{1 + |x_i - y_i|}}\]
	(注:利用$\frac{|a + b|}{1 + |a + b|} \leqslant \frac{|a| + |b|}{1 + |a| + |b|} \leqslant \frac{|a|}{1 + |a|} + \frac{|b|}{1 + |b|}$证明三角不等式.)
	\item 可测函数空间$M(E)$.$m(E) < \infty$,$M(E)$是$E$上可测函数的全体,将$\aev$相等的两个函数记为同一元.$\forall x = x(t), y = y(t)$,\[d(x, y) = \int_E \frac{|x(t) - y(t)|}{1 + |x(t) - y(t)|} \dif t\]
	(注:利用$d(x, y) = 0 \iff x(t) = y(t) \aev$证明非负性.)
	\item 离散距离空间.$X \neq \emptyset, \forall x, y \in X$,
	\begin{equation*}
		d(x, y) = 
		\begin{cases}
			0, \quad x = y \\
			1, \quad x \neq y
		\end{cases}
	\end{equation*}
	(注:这说明任意空间都可以定义距离,但是可能没有实际意义.)
\end{itemize}

距离空间上集合的距离:$X$是距离空间,$A, B \subset X, d(A, B) = \inf{d(x, y): x \in A, y \in B}$.

有界集:$A \subset X, \exists x_0 \in X, M > 0, \st \forall x \in A, d(x, x_0) \leqslant M$.

序列的极限:$\{x_n\}$是距离空间$X$的序列,$x \in X$,若$\lim_{n \to \infty}{d(x_n, x)} = 0$,则称$\{x_n\}$按距离收敛于$x$,记为$\lim_{n \to \infty}{x_n} = x$.

注:不同空间中的序列在不同意义下的收敛,通过定义适当的距离,可以归结为距离空间中序列的按距离收敛.

距离空间的性质(根据数学分析的思路证明):
\begin{itemize}[itemindent=2em]
	\item 收敛序列的极限是唯一的;
	\item 收敛序列是有界的;
	\item 收敛序列的任一子列仍收敛于同一极限;
	\item 距离函数是两个变元的连续函数,即$x_n \to x, y_n \to y, d(x_n, y_n) \to d(x, y)$.
\end{itemize}

(第四条性质的证明如下:$\forall x, y, z \in X, d(x, y) \leqslant d(x, z) + d(z, y)$,则$d(z, y) - d(x, y) \leqslant d(z, x) = d(x, z)$.因此$|d(x, y) - d(z, y)| \leqslant d(x, z)$.这是减法不等式,故有:$|d(x_n, y_n) - d(x, y)| \leqslant |d(x_n, y_n) - d(x, y_n)| + |d(x, y_n) - d(x, y)| \leqslant d(x_n, x) + d(y_n, y) \leqslant 0.$)

\section{赋范空间的基本概念}
线性空间:$X$是一个非空集,$\mathbb{K}^n$是标量域.$X$是线性空间,若
\begin{itemize}[itemindent=2em]
	\item $X$上加法满足交换律,结合律,存在唯一的零元,存在唯一的逆元(负元).
	\item $X$上数乘满足结合律,与加法共同满足左右交换率,且$1x = x$.
\end{itemize}

线性子空间:$E \subset X$,且$E$对$X$上的加法运算和数乘运算封闭,则$E$是$X$的线性子空间.

线性相关,线性无关,线性空间的维数,(注意无穷维线性空间,线性无关向量组的个数可以任意大),线性空间的基与坐标.

线性算子:$T:X \to Y, \forall x_1, x_2 \in X, \alpha, \beta \in \mathbb{K}, T(\alpha x_1 + \beta x_2) = \alpha T(x_1) + \beta T(x_2)$.

由$E$张成的线性子空间:\[\mathrm{span}(E) = \{\sum_{i = 1}^{n}{\lambda_i x_i: x_1, x_2, \cdots, x_n \in E, \lambda_1, \lambda_2, \cdots, \lambda_n \in \mathbb{K}, n = 1, 2, \cdots}\}\].

线性空间的(抽象)例子:
\begin{itemize}[itemindent=2em]
	\item 数列空间:$s$是实或复数列的全体.定义的加法为按坐标相加,数乘为按坐标数乘.
	\item 函数空间:$X$是实或复函数的全体.$x, y \in X, \alpha \in \mathbb{K}$,$(x + y)(t) = x(t) + y(t), (\alpha x)(t) = \alpha x(t)$.
\end{itemize}

倍集与线性和集:$\lambda A = {\lambda x: x \in A}, A + B = {x + y:x \in A, y \in B}$.(集合可以为单点集)

范数$\forall x \in X, \exists \|x\| \in \mathbb{R}$,满足:
\begin{itemize}[itemindent=2em]
	\item 正定性:$\|x\| \geqslant 0, \| x \| = 0 \iff x = 0$;
	\item 绝对奇性:$\| \alpha x \| = |\alpha| \| x \|, x \in X, \alpha \in \mathbb{K}$;
	\item 三角不等式:$\| x + y \| \leqslant \| x \| + \| y \|$.
\end{itemize}

此外还有:$|\norm{x} - \norm{y}| \leqslant \norm{x - y}$.

由范数导出的距离:$d(x, y) = \norm{x - y}$.

按范数收敛:$\norm{x_n - x} \to 0(n \to \infty)$,则称$\{x_n\}$按范数收敛于$x$.记号与按距离收敛一致.

\textbf{距离可由范数导出的充要条件:}
\begin{itemize}[itemindent=2em]
	\item $d(\alpha x, 0) = |\alpha| d(x, 0)$;
	\item $d(x + z, y + z) = d(x, y)$.
\end{itemize}

范数的连续性:
\begin{itemize}[itemindent=2em]
	\item (1)$\norm{x}$ 是$X$上的连续函数.即$x_n \to x, \norm{x_n} \to \norm{x}$;
	\item (2)$X$上的加法与数乘运算是连续的.$\forall \{x_n\}, \{y_n\} \subset X, \{\alpha_n\} \subset \mathbb{K}, x_n \to x, y_n \to y, \alpha_n \to \alpha$,$x_n + y_n \in x + y, \alpha_n x_n \to \alpha x$.
\end{itemize}

(注:(1)由范数的减法不等式得到,(2)加法的连续性由范数的减法不等式得到,数乘的连续性先需要说明$\{\alpha_n\}$的有界性,然后将式子分解后利用减法不等式与范数的绝对奇性:$\norm{\alpha_n x_n - \alpha x} \leqslant |\alpha_n| \norm{x_n - x} + |\alpha_n - \alpha| \norm{x}$)

赋范空间的例子:
\begin{itemize}[itemindent=2em]
	\item 欧式空间$\mathbb{K}^n$.可以定义很多范数.
	\begin{align*}
		\nnorm{x}{2} & = (\sum_{i = 1}^{n}{x_i})^\frac{1}{2} \\
		\nnorm{x}{1} & = \sum_{i = 1}^{n}{x_i} \\
		\nnorm{x}{\infty} & = \max_{1 \leqslant i \leqslant n}{|x_n|}
	\end{align*}.
	\item $C[a, b]$:\[\norm{x} = \max_{a \leqslant t \leqslant b}{|x(t)|}.\]
	\item $c, c_0$,收敛的实或复数列的全体,$x = (x_1, x_2, \cdots) \in c$,\[\norm{x} = \sup_{i \geqslant 1}{|x_i|}.\] $c_0$是收敛于$0$的数列全体,范数定义与$c$上范数定义相同.
	\item Lebesgue可积函数空间$L[a, b]$:将$\aev$相等的两个函数记为同一元.\[\nnorm{f}{1} = \int_{a}^{b}|f| \dif x.\]
\end{itemize}

赋范空间的线性子空间也是赋范空间.

\section{$L^p$空间与$l^p$空间}
\subsection{$L^p, 1 \leqslant p < \infty$}
定义:$E$上的$p$次方可积函数的全体称为$L^p(E)$.

由于$|f + g|^p \leqslant 2^p(|f|^p + |g|^p), (x \in E)$,$f \in L^p(E) \Rightarrow \alpha f \in L^p(E)$.$L^p(E)$是线性空间.

范数:\[\nnorm{f}{p} = (\int_E |f|^p \dif x)^\frac{1}{p}.\]

验证范数的合理性(良定义,此处只验证三角不等式):

Young不等式:$\forall a, b \geqslant 0, a, b \in \mathbb{R}, 1 < p, q < \infty, \frac{1}{p} + \frac{1}{q} = 1, ab \leqslant \frac{a^p}{p} + \frac{b^q}{q}$.

(证明利用$\varphi(x) = \ln{x}$的上凸性质.$\lambda \ln{x} + (1 - \lambda)\ln{y} \leqslant \ln{\lambda x + (1 - \lambda) y}$,令$\lambda = \frac{1}{p}, x = a^p, y = b^p$)

\textbf{Holder不等式}:$p, q$条件如上(共轭指标),$f \in L^p(E), g \in L^q(E)$, 则$fg \in L^1(E)$,且$\nnorm{fg}{1} \leqslant \nnorm{f}{p}\nnorm{g}{q}$.

(证明使用:$a = \frac{|f(x)|}{\nnorm{f}{p}}, b = \frac{|g(x)|}{\nnorm{g}{q}}$,对Young不等式两边分别积分.注意存在范数为零的情况单独讨论.)

Minkowski不等式($L^p$空间范数的三角不等式):$1 \leqslant \infty, f, g \in L^p(E)$,则$\nnorm{f + g}{p} \leqslant \nnorm{f}{p} + \nnorm{g}{p}$.

(证明先拆出一个$|f + g|$,然后分解为$|f| + |g|$,对这些部分分别使用Holder不等式,合并即可.注意存在范数为零的情况,需要单独讨论.)

按范数收敛:$f_n \stackrel{L^p}{\longrightarrow} f, (n \to \infty)$.\[\frac{1}{m(E)}\int_E |f_n - f|^p \dif x = \frac{1}{m(E)} \nnorm{f_n - f}{p}^p \to 0 (n \to \infty)\]

按范数收敛能够推出按测度收敛.

证明:$\nnorm{f_n - f}{p} \to 0$,则$\forall \varepsilon > 0, n \to \infty$,有
\begin{align*}
	m(E(|f_n - f| > \varepsilon)) & = m(E(|f_n - f|^p > \varepsilon^p)) \leqslant \frac{1}{\varepsilon^p} \int_{E} |f_n - f|^p \dif x \\
								  & = \frac{1}{\varepsilon^p} \nnorm{f_n - f}{p} \to 0.
\end{align*}

由Risez定理可知,$f_n \stackrel{L^p}{\longrightarrow} f$,则$\exists \{f_{n_k}\} \subset \{f_n\}, \st f_{n_k} \to f \aev$

\subsection{$L^{\infty}(E)$}
本性有界:$\exists M > 0, \st |f| \leqslant M \aev \mbox{于}E$.

范数:\[\nnorm{f}{\infty} = \inf\{M:|f| \leqslant M \aev\}.\]

验证范数的合理性(良定义,此处只验证三角不等式):

$\forall n \geqslant 1, n \in \mathbb{N}, \exists E_n \subset E, m(E_n) = 0$,并且\[|f(x)| \leqslant \nnorm{f}{\infty} + \frac{1}{n} (x \in E \backslash E_n).\]

令$E_0 = \bigcup_{n = 1}^{\infty}{E_n}$,则$m(E_0) = 0$,则$\forall n \geqslant 1, n \in \mathbb{N}, \exists E_n \subset E, m(E_0) = 0$,令$n \to \infty$得到$|f(x)| \leqslant \nnorm{f}{\infty}(x \in E \backslash E_0).$,则$|f| \leqslant \nnorm{f}{\infty} \aev$.

$|f + g| \leqslant |f| + |g| \leqslant \nnorm{f}{\infty} + \nnorm{g}{\infty} \aev$,因此$\nnorm{f + g}{\infty} \leqslant \nnorm{f}{\infty} + \nnorm{g}{\infty}$.

与$L^p$空间的关系:$m(E) < \infty$,则$\lim_{p \to \infty}\nnorm{f}{p} = \nnorm{f}{\infty}$.(证明略)

Holder不等式在共轭指标$p = 1, q = \infty$或$p = \infty, q = 1$时均成立.(将无穷范数单独计算后提出积分式即可)

\subsection{$l^p, 1 \leqslant p < \infty$}
定义:$x = (x_n)$是实或复数列.若$\sum_{i = 1}^{\infty}{|x_n|^p} < \infty$,则称$x$是$p$次方可和的.$p$次方可和函数的全体称为$l^p$.

范数:\[\nnorm{x}{p} = (\sum_{i = 1}^{\infty}{|x_n|^p)})^\frac{1}{p}\]

同样有Holder不等式与Minkowski不等式,条件与结论与之前一致.按范数定义替代之即可.

\subsection{$l^{\infty}$}
定义:$l^{\infty}$是有界数列的全体.

范数:$\forall x = (x_n) \in l^{\infty}$,\[\nnorm{x}{\infty} = \sup_{n \geqslant 1}{|x_n|}.\]

Minkowski不等式在$0 < p < 1$上不成立.(实际上是反向)

\section{点集,连续映射与可分性}
内点,开集,内部,聚点,导集,闭集,闭包,开集的基本性质,导集的等价定义,闭包的等价定义,闭集的等价定义,稠密集,稠密集的等价定义.(注意邻域语言与序列语言的表达,这些定义在实变第一章第六节中均有提及),以下补充疏朗集的等价定义.

\textbf{疏朗集的等价定义(TFAE):}($A \subset X$)
\begin{itemize}[itemindent=2em]
	\item $A$是疏朗集.
	\item 对任一开球$U$,$\exists U_1 \subset U, \st U_1 \cap A = \emptyset$.
	\item 对任一闭球$S$,$\exists S_1 \subset S, \st S_1 \cap A = \emptyset$.
\end{itemize}

(证明:$A$是疏朗集等价于对任一开球$U(x, \varepsilon)$,包含关系$U(x, \varepsilon) \subset \bar{A}$不成立,即存在$y \in U(x, \varepsilon), \st y \neq \bar{A}$,这又等价于$\exists U(y, \delta) \subset U(x, \varepsilon) \st U(y, \delta) \cap A = \emptyset$,由于每个开球中包含一个闭球,反之亦然,后两条性质等价.)

映射,单射,满射,双射,像,原像.

\textbf{连续映射}:$X, Y$是距离空间,$T:X \to Y, x_0 \in X, \forall \varepsilon > 0, \exists \delta > 0, \st d(x, x_0) < \delta, d(Tx, Tx_0) < \varepsilon$,则称$T$在点$x_0$处连续,若$T$在$X$上的每一点处连续,则称$T$在$X$上连续.

邻域语言的等价描述:对于$Tx_0$的任一邻域$V$,存在$x_0$的邻域$U$,使得$T(U) \subset V$,则称$T$在$X$上连续.

连续的充要条件:$X, Y$是距离空间,$T:X \to Y, x_0 \in X$,则$T$在$x_0$处连续的充要条件为:$\forall \{x_n\} \subset X, x_n \to x_0, Tx_n \to Tx_0$.(必要性:直接使用$\varepsilon-N$语言证明即可,充分性:用反证法,先用$\varepsilon-N$语言把条件写出来,然后根据$x_n \to x_0, Tx_n \nrightarrow Tx_0$推出矛盾)

\textbf{连续映射的等价条件(TFAE):}
\begin{itemize}
	\item $T$在$X$上连续.
	\item $G \subset Y$,$G$为开集,$T^{-1}(G)$是$X$中的开集.
	\item $F \subset Y$,$F$为闭集,$T^{-1}(F)$是$X$中的闭集.
\end{itemize}

证明:$(1) \Rightarrow (2).$设$G$是$Y$中的开集,不妨设$T^{-1}(G) \neq \emptyset$.$\forall x_0 \in T^{-1}(G)$,由于$Tx_0 \in G$,并且$G$是开集,因此存在$Tx_0$的一个邻域$V \subset G$,由$T$的连续性,$\exists U(x_0), \st T(U) \subset V \subset G$.于是$U \subset T^{-1}(G)$.因此$x_0$是$T^{-1}(G)$的内点,这表明$T^{-1}(G)$是开集.
$(2) \Rightarrow (1)$.设$x_0 \in X$,$V$是$Tx_0$的一个邻域.由于$V$是$Y$中的开集,$T^{-1}(V)$是开集,$x_0 \in T^{-1}(V)$.$\exists U(x_0), \st T(U) \subset V $.则$T$在$x_0$处连续,由$x_0$的任意性得证.
$(2) \Leftrightarrow (3)$.利用$\forall A \subset Y, T^{-1}(A^C) = {(T^{-1}(A))}^C$.

空间的可分性:设$X$是距离空间,若在$X$中存在一个\textbf{可数的稠密子集},则称$X$是可分的.

\textbf{$l^p$是可分的},$A = {(r_1, r_2, \cdots, r_n, 0, \cdots): r_i \in \mathbb{Q}, n = 1, 2, \cdots}$(可列集).利用有理数在实数集中的稠密性与收敛数列的余项趋于0这两个性质,控制两个元之间的距离.

\textbf{$C[a, b]$是可分的},$P_0[a, b]$是有理系数多项式的全体(可列集),利用Weierstrass逼近定理,将$[a, b]$上的连续函数用多项式一致逼近,然后用有理系数多项式逼近普通的多项式,控制两个元之间的距离.

\textbf{$L^p[a, b], 1 \leqslant p < \infty$是可分的},$P_0[a, b]$是有理系数多项式的全体(可列集),首先用简单函数列逼近,然后使用控制收敛定理,用一个简单函数$g$逼近$x$,利用Lusin定理,用连续函数$h$逼近$g$,利用上一部分的结论,使用有理系数多项式$p_0$逼近$h$,控制两个元之间的距离.

\textbf{$l^{\infty}$不是可分的}.令$K = {x = (x_1, x_2, \cdots): x_i = 0 \mbox{或} 1}$,则$K$是不可数集(二元数列),$\forall x, y \in K, d(x, y) = 1$.$A$是$l^{\infty}$的可数子集.若$A$在$l^{\infty}$中是稠密的,$U(x, \frac{1}{3})$中至少包含$A$中的一个元,由于开球$U(x, \frac{1}{3})$有不可数个,$\exists z \in A, z \in U(x, \frac{1}{3}) \cap U(y, \frac{1}{3})$,此时$d(x, y) \leqslant \frac{2}{3} \neq 1$,矛盾.

可列集的意义:在可分的空间$X$中选取一个适当的可数的稠密子集,在这个集合上研究,然后取一个极限过程,得到全空间$X$上的相应结论.

\section{完备性}
Cauchy序列:$\{x_n\} \subset X, \forall (\mbox{给定的}) \varepsilon > 0, \exists N > 0, \st m, n > N, d(x_m, x_n) < \varepsilon$.

Cauchy序列的性质:
\begin{itemize}[itemindent=2em]
	\item 收敛序列是Cauchy序列.
	\item Cauchy序列是有界的.
	\item $\{x_n\}$是Cauchy序列,并且存在一个子列$\{x_{n_k}\}$收敛于$x$,则$\{x_n\}$收敛于$x$.
\end{itemize}

完备空间:$X$是距离空间,若$X$中的每个Cauchy序列都是收敛的,则称$X$是完备的.完备的赋范空间称为Banach空间.

$P[a, b]$不是完备的.$P[a, b] \subset C[a, b]$.令$p_n(t) = 1 + \frac{t}{1!} + \frac{t^2}{2!} + \cdots + \frac{t^n}{n!}, n = 1, 2, \cdots$,则$p_n \in P[a, b]$,又有$p_n \to e^t$.(利用Taylor展开),因此$\{p_n\}$是Cauchy序列,但$e^t \not\in P[a, b]$.这表明$\{p_n\}$在$P[a, b]$中不收敛.

$l^p$是完备的.(仅证明$p \neq \infty$的部分,可以用作证明完备性的模板)

设$x^{(n)} = (x_1^{(n)}, x_2^{(n)}, \cdots)$是$l^p$中,的Cauchy序列.则$\forall \varepsilon > 0, \exists N > 0, \st m, n > N$,\[\sum_{i = 0}^{\infty}{|x_i^{(m)} - x_i^{(n)}|^p} = \nnorm{x^{(m)} - x^{(n)}}{p}^p < \varepsilon^p.\]

于是对每个\textbf{固定的}$i$,当$m, n > N$,\textbf{($N$固定)},\[|x_i^{(m)} - x_i^{(n)}| \leqslant \nnorm{x^{(m)} - x^{(n)}}{p} < \varepsilon. (*)\]

这表明对每个\textbf{固定的}$i$,$\{x_i^{(n)}\}_{n \geqslant 1}$是Cauchy序列,因此$\{x_i^{n}\}$收敛,设当$n \to \infty$时,$x_i^{(n)} \to x_i, i = 1, 2, \cdots$,令$x = (x_1, x_2, \cdots)$.下面证明$x \in l^p$并且$x^{(n)} \to x$.(注:从按坐标收敛到按范数收敛)

由$(*)$式得到,$\forall k \geqslant 1$,当$m, n > N$时,\[\sum_{i = 1}^{k}|x_i^{(m)} - x_i^{(n)}|^p < \varepsilon^p.\]

\textbf{固定$N$},先令$m \to \infty$,再令$k \to \infty$,得到\[\sum_{i = 1}^{\infty}{|x_i - x_i^{(n)}|^p} \leqslant \varepsilon^p.\]

这表明,$x - x^{(n)} \in l^p$,由于$l^p$是线性空间,$x = (x - x^{(n)}) + x^{(n)} \in l^p$,且有$\nnorm{x - x^{(n)}}{p} \leqslant \varepsilon.$因此$x^{(n)} \to x, n \to \infty.$

$c_0, c, l^{\infty}$均是完备的.

$C[a, b]$是完备的.设$\{x_n\}$是$C[a, b]$中的Cauchy序列,则$\forall \varepsilon > 0, \exists N > 0, \st m, n > N, \norm{x_m - x_n} < \varepsilon$.于是$\forall t \in [a, b]$,当$m, n > N$时,\[|x_m(t) - x_n(t)| \leqslant \max_{a \leqslant t \leqslant b}{|x_m(t) - x_n(t)|} = \norm{x_m - x_n} < \varepsilon.\]

这表明对每个\textbf{固定的}$t$,$\{x_n(t)\}_{n \geqslant 1}$是Cauchy序列.令$x(t) = \lim_{n \to \infty}{x_n(t)}, t \in [a, b]$.

\textbf{固定$N$},令$m \to \infty$.得到$|x(t) - x_n(t)| < \varepsilon$.则$x_n(t) \rightrightarrows x(t), x = x(t)$,且$\norm{x - x_n} = \max_{a \leqslant t \leqslant b}{|x(t) - x_n(t)| \leqslant \varepsilon}$,故$x_n \to x$.

$L^p(E), (1 \leqslant p \leqslant \infty)$均是完备的.(证明略)

纲集:$X$是距离空间,$A \subset X$,若$A$可以表示成为可数个疏朗集的并,则$A$是第一纲集,否则$A$是第二纲集.

\textbf{完备空间的性质:}
\begin{itemize}[itemindent=2em]
	\item (闭球套定理)设$X$是完备的距离空间,$S_n = {x:d(x, x_n) \leqslant r_n}(n = 1, 2, \cdots)$是$X$中的一列闭球,满足$S_{n + 1} \subset S_n,n \geqslant 1$且$S_n$的半径$r_n \to 0$,则$\exists|x \in \bigcap_{n = 1}^{\infty}{S_n}$.(证明与数分类似,先证明$\{x_n\}$是Cauchy序列,利用完备性与闭集性质,唯一性可以设存在另一个点$x'$,证明$d(x, x') = 0$)
	\item (Baire纲定理)完备的距离空间是第二纲集.(证明使用反证法,将$X$表示成为可数个疏朗集的并,$X = \bigcup_{n = 1}^{\infty}{A_n}$,根据疏朗集性质构造闭球套,由$X$的完备性,存在一个$x$属于这一些疏朗集的交,但是$S_n \cap A_n = \emptyset$,从而导致矛盾)
\end{itemize}

级数,部分和,绝对收敛.(了解即可)

不动点:$X$是距离空间,$T:X \to X, \exists x_0 \in X, \st Tx_0 = x_0$,则称$x_0$为$T$的一个不动点.

\textbf{压缩映射}:$X$是距离空间,$T:X \to X$,$\exists 0 \leqslant \lambda < 1, \st d(Tx, Ty) \leqslant \lambda d(x, y), x, y \in X$.称$T$是压缩的.

压缩映射是连续的.

\textbf{压缩映射原理:完备距离空间上的压缩映射存在唯一的不动点.}

证明:任取$x_0 \in X$,令$x_1 = Tx_0, x_2 =Tx_1, \cdots, x_n = Tx_{n - 1}, \cdots$.

$d(x_{n + 1}, x_n) = d(Tx_n, Tx_{n - 1}) \leqslant \lambda d(x_n, x_n - 1) \leqslant \cdots \leqslant \lambda^n d(x_1, x_0)$.于是$\forall n, p \in \mathbb{N}$,

\begin{align*}
	d(x_{n + p}, x_n) & \leqslant d(x_{n + p}, x_{n + p - 1}) + d(x_{n + p - 1}, x_{n + p - 2}) + \cdots + d(x_{n + 1}, x_n) \\
					  & \leqslant (\lambda^{n + p - 1} + \lambda^{n + p - 2} + \cdots + \lambda^{n}) d(x_1, x_0) \\
					  & = \lambda^n(\lambda^{p - 1} + \lambda^{p - 2} + \cdots + 1)d(x_1, x_0) \\
					  & \leqslant \frac{\lambda^n}{1 - \lambda} d(Tx_0, x_0).
\end{align*}

则$\{x_n\}$是$X$中的Cauchy序列.由于$X$是完备的,$\{x_n\}$收敛,设$\lim_{n \to \infty}{x_n} = x$.由$T$的连续性得到$x = \lim_{n \to \infty}{x_n} = \lim_{n \to \infty}{Tx_{n - 1}} = Tx$.存在性得证.

若另有$y \in X, \st Ty = y$,则$d(x, y) = d(Tx, Ty) \leqslant \lambda d(x, y) = 0.$唯一性得证.

注:$x_0$经过$T$的$n$次迭代后与$x$的距离估计:$d(x, x_n) \leqslant \frac{\lambda^n}{1 - \lambda} d(Tx_0, x_0)$.

注:压缩条件减弱至$d(Tx, Ty) < d(x, y)$时,若$X$是闭集,压缩映射原理仍成立,否则将不再成立.

应用(微分方程中的Picard定理):微分方程:
\begin{equation*}
	\frac{\dif x}{\dif y} = f(t, x), x(t_0) = x_0, f \in C(\mathbb{R}^2), \mbox{满足Lipschitz条件}.
\end{equation*}
则在$t_0$的某邻域内,微分方程有唯一的解.(在$C[a, b]$上考虑,将微分方程转化为积分方程,做出映射,并利用Lipschitz条件证明它是压缩的即可,注意要将$\norm{x_1 - x_2}$放缩出来)

等距同构:$X, Y$是距离空间,若存在映射$T:X \to Y, \st T$是双射,并且对任意$x_1, x_2 \in X$,有$d(Tx_1, Tx_2) = d(x_1, x_2)$,则称$X, Y$是等距同构的.

完备化空间:设$(X, d)$是距离空间,若存在完备的距离空间$(\tilde{X}, \tilde{d})$,使得$X$与$\tilde{X}$的一个稠密子空间等距同构,则称$\tilde{X}$是$X$的完备化空间.

\textbf{若$\tilde{X}$是一个完备的距离空间,使得$X \subset \tilde{X}$,并且$X$在$\tilde{X}$中稠密,则$\tilde{X}$是$X$的完备化空间.}

拓扑同构:存在$T:X \to Y$,使得$T$是双射和线性的,并且$\exists a, b > 0, \st a \norm{x} \leqslant \norm{Tx} \leqslant b \norm{x},x \in X$,则称$X, Y$是拓扑同构的.

等距同构:存在$T:X \to Y$,使得$T$是双射和线性的,并且$\norm{Tx} = \norm{x}, x \in X$,则称$X, Y$是等距同构的.

等距同构可以推出拓扑同构.

范数的等价性:设$X$是线性空间,$\nnorm{\cdot}{1}, \nnorm{\cdot}{2}$是$X$上的两个范数,称$\nnorm{\cdot}{1}, \nnorm{\cdot}{2}$是等价的,若存在常数$a, b > 0, \st a \nnorm{x}{1} \leqslant \nnorm{x}{2} \leqslant b \nnorm{x}{1}, x \in X$.

两个范数等价时,以它们为范数构成的赋范空间是拓扑同构的.

设$(X, \norm{\cdot})$是$n$维赋范空间,$e_1, e_2, \cdots, e_n$是$X$的一组基,则存在常数$a, b > 0, \st \forall x = \sum_{i = 1}^{n}{x_i e_i}$,有
\begin{equation*}
	a(\sum_{i = 1}^{n}{|x_i|^2})^\frac{1}{2} \leqslant \norm{x} \leqslant b (\sum_{i = 1}^{n}{|x_i|^2})^\frac{1}{2}.
\end{equation*}
(证明略)

有限维赋范空间的性质:
\begin{itemize}[itemindent=2em]
	\item 有限维赋范空间上的任意两个范数都是等价的.
	\item 任意$n$维赋范空间与$\mathbb{K}^n$拓扑同构.
	\item 有限维赋范空间是完备的,任何赋范空间的有限维子空间都是闭子空间.
\end{itemize}

证明:(1)利用范数的等价性.

(2)取一组基,做映射$T:X \to K^{n}, T(\sum_{i = 1}^{n}{x_i e_i}) = (x_1, x_2, \cdots, x_n)$,在$a\norm{x} \leqslant \norm{Tx} b\norm{x}, x \in X$,由$\frac{1}{b}{\norm{x}} \leqslant \norm{Tx} = (\sum_{i = 1}^{n}{|x_i|^2})^\frac{1}{2} \leqslant \frac{1}{a}\norm{x}$.

(3)由(2)可知$X$与$\mathbb{K}^n$拓扑同构,$\exists T:X \to \mathbb{K}^n, T$是双射与线性的,$\exists a, b > 0, \st a \norm{x} \leqslant \norm{Tx} \leqslant b \norm{x}, x \in X$,由$\{x_n\}$是Cauchy序列,$\forall m, n \in \mathbb{K}^n$,

$\norm{Tx_m - Tx_n} = \norm{T(x_m - x_n)} \leqslant b \norm{x_m - x_n}$,$\{Tx_n\}$是Cauchy序列,$\exists y, Tx_n \to y$,$T$是满射,$\exists x \in X \st Tx = y$,于是$\norm{x_n - x} \leqslant \frac{1}{a} \norm{T(x_n - x)} = \frac{1}{a} \norm{Tx_n - y} \to 0, n \to \infty$,因此$x_n \to x.$这就证明$X$是完备的.

\section{紧性}
开覆盖:设$X$是距离空间,$A \subset X$.若$\{G_\alpha, \alpha \in I\}$是$X$中的一族开集,使得$\bigcup_{\alpha \in I}{G_{\alpha} \supset A}$,则称$\{G_{\alpha}$是$A$的一个开覆盖.

\textbf{紧集:}对于$A$的任一开覆盖$\{G_{\alpha}, \alpha \in I\}$都存在其中的有限个开集仍覆盖$A$,则称$A$是紧集.(对应有限覆盖性质)

\textbf{列紧集:}若$A$中任一序列都存在收敛的子列,则称$A$是列紧集.(序列性质)

\textbf{完全有界集:}若对$\forall \varepsilon > 0, \exists \mbox{有限集} E = {x_1, x_2, \cdots, x_n}, \st \bigcup_{i = 1}^{k}U(x_i, \varepsilon) \supset A$,则称$A$是完全有界集,称$E$为$A$的有限$\varepsilon-$网.

\textbf{完全有界集是有界集.}

(取$\varepsilon = 1, \exists x_1, x_2, \cdots, x_k \in X, \st \bigcup_{i = 1}^{k}{U(x_i, 1) \supset A}$,$\forall x \in A, \exists 1 \leqslant i \leqslant k, \st x \in U(x_i, 1)$,于是$d(x, x_1) \leqslant d(x, x_i) + d(x_i, x_1) < 1 + \max_{1 \leqslant i \leqslant k}{d(x_i, x_1)}.$)

\textbf{$A \subset X$,$A$是完全有界集的充要条件是$A$中的任一序列必存在Cauchy子列.}(完全有界集的序列语言,证明略)

\textbf{列紧集是完全有界集;若$X$是完备的,则完全有界集是列紧集.}

\textbf{$A \subset X$,$A$是紧集的充要条件是$A$中的任一序列必存在收敛于$A$中元的子列.}(紧集的序列语言,证明略)

\textbf{紧集是列紧集.}

\textbf{紧集是有界闭集.}(证明:$A$是紧集,则$A$是列紧集,则$A$是完全有界集,$A$是有界集.$\{x_n\}$是$A$中的序列$x_n \to x$,则$\exists {x_{n_k}} \subset {x_n}, \st x_{n_k} \to y$,而$x_n \to x$,则$x = y \in A$,这表明$A$是闭集.)

\textbf{在有限维赋范空间中,有界集是列紧集,有界闭集是紧集.}

证明:设$X$是$n$维赋范空间.$X$与$\mathbb{K}^n$拓扑同构,即存在$T:X \to \mathbb{K}^n, \st T$是双射和线性的,并且存在常数$a, b > 0, \st a \norm{x} \leqslant \norm{Tx} \leqslant b \norm{x}, x \in X$,设$A$是$X$中的有界集,$\norm{x} \leqslant M, x \in A$.设$\{x_n\} \subset A$,得到$\norm{Tx_n} \leqslant b \norm{x_n} \leqslant bM, n \leqslant 1$,则$\{Tx_n\}$是$\mathbb{K}^n$的有界序列,根据Weierstrass定理,$\exists \{Tx_{n_k}\} \subset \{Tx_n\}, y \in \mathbb{K}^n, \st \norm{Tx_{n_k} - y} \to 0$,设$x \in X, \st Tx = y.$得到$\norm{x_{n_k} - x} \leqslant \frac{1}{a} \norm{T(x_{n_k}- x)} = \frac{1}{a} \norm{Tx_{n_k} - y} \to 0, n \to \infty.$故$x_{n_k} \to x.$

再设$A$是有界闭集,$A$是列紧集,故$A$是紧集.

连续函数的性质:设$A$是距离空间$X$中的紧集,$f$是$A$上的连续的实值函数.则$f$在$A$上有界,且在$A$上达到上下确界.

证明:有界性.若$f$在$A$上无上界,则$\forall n \in \mathbb{N}, \exists x_n \in A, \st f(x_n) > n$.由于$A$是紧集,$\exists \{x_{n_k}\} \subset {x_n}, \st x_{n_k} \to x \in A.$由于$f \in C(A), f(x_{n_k}) \to f(x).$但$f(x_{n_k}) > n_k \to \infty$,矛盾.有界性得证.记$a = \sup_{x \in A}{f(x)}$,则$\exists y_n \in A \st f(y_n) \to a$.由于$A$是紧集,$\exists \{y_{n_k}\} \subset y \in A$,又$f$的连续性得到$f(y) = \lim_{k \to \infty}{f(y_{n_k})} = a$.

\section{有界线性算子的基本概念}
线性算子,线性泛函.

$T$是线性算子,则$N(T) = {x \in X: Tx = 0}$为$T$的零空间,$R(T) = {Tx: x \in X}$为$T$的值域.它们分别是$X, Y$的线性子空间.

线性算子和线性泛函的例子:

$X$是$n$维线性空间,$e_1, e_2, \cdots, e_n$是$X$的一组基,$A = (a_{ij})$是一个$n \times n$阶矩阵,定义算子:\[T:X \to X, T(\sum_{j = 1}^{n}{x_j e_j}) = \sum_{i = 1}^{n}{y_i e_i}, y_i = \sum_{j = 1}^{n}{a_{ij} x_j}, i = 1, 2, \cdots, n.\]

则$T$是线性算子,反过来,设$T:X \to X$是线性算子,则$Te_j$必是$e_1,e_2,\cdots,e_n$的线性组合,设$Te_j = \sum_{i = 1}^{n}{a_{ij} e_i}, j = 1, 2, \cdots, n.$当$x = \sum_{j = 1}^{n}{x_j e_j}$时,
\begin{align*}
	y & = Tx = \sum_{i = 1}^{n}{y_j e_i} = \sum_{j = 1}^{n}{x_j T e_j} \\
	  & = \sum_{j = 1}^{n}{x_j} \sum_{i = 1}^{n}{a_{ij}e_i} = \sum_{i = 1}^{n}{\sum_{j = 1}^{n}{a_{ij} x_j } e_i }
\end{align*}

则$X$上的线性算子与$n \times n$阶矩阵一一对应.

定义泛函:\[f(x) = \sum_{i = 1}^{n}{a_i x_i}\]

则$f$是$X$上的线性泛函,记$a_i = f(e_i)$,则\[f(x) = f(\sum_{i = 1}^{n}{x_i e_i}) = \sum_{i = 1}^{n}{x_i}{f(e_i)} = \sum_{i = 1}^{n}{a_i x_i}.\]

则$X$上的线性泛函与$\mathbb{K}^n$中的向量一一对应.

设$K(s, t)$是$[a, b] \times [a, b]$上的可测函数,满足$M = (\int_{a}^{b}(\int_{a}^{b}|K(s, t)|^2 \dif t) \dif s)^\frac{1}{2} < \infty$.对任意$x \in L^2[a, b]$,令
\begin{align*}
	(Tx) (s) & = \int_{a}^{b} K(s, t) x(t) \dif t \\
	f(x)	 & = \int_{a}^{b} x(t) \dif t.
\end{align*}

分别为$L^2[a, b]$上的线性算子(第二类Fredholm积分算子)和线性泛函.由Holder不等式,
\begin{align*}
	\int_{a}^{b} |(Tx) (s)|^2 \dif s & = \int_{a}^{b}|\int_{a}^{b} K(s, t) x(t) \dif t|^2 \dif s \\
									 & \leqslant \int_{a}^{b}(\int_{a}^{b}|K(s, t)|^2 \dif t \int_{a}^{b} |x(t)|^2 \dif t) \dif s \\
									 & = \int_{a}^{b}(\int_{a}^{b}|K(s, t)|^2 \dif t) \dif s \int_{a}^{b} |x(t)|^2 \dif t \\
									 & = M^2 \nnorm{x}{2}^2 < \infty.
\end{align*}

故$Tx \in L^2[a, b]$.由积分的线性性知道$T$是线性的.同样由Holder不等式,

\begin{equation*}
	\int_{a}^{b} |x(t)| \dif t \leqslant (\int_{a}^{b} 1 \dif t)^\frac{1}{2}(\int_{a}^{b}|x(t)|^2 \dif t)^\frac{1}{2} = \sqrt{b - a} \nnorm{x}{2}^2.
\end{equation*}

故$f$是$L^2[a, b]$上的线性泛函.

算子的有界性:$X, Y$是赋范空间,$T:X \to Y$是线性算子,若$T$将$X$中的每个有界集都映射为$Y$中的有界集,则称$T$是有界的.

\textbf{有界线性算子的等价定义(TFAE):}
\begin{itemize}[itemindent=2em]
	\item (1)$T$是有界的;
	\item (2)$\exists c > 0, \st \norm{Tx} \leqslant c \norm{x}, x \in X$;
	\item (3)$T$在$X$上连续.
\end{itemize}

证明:$(1) \Rightarrow (2)$.由于$S = {x: \norm{x} \leqslant 1}$是$X$中的有界集,$T$是有界的,$T(S)$是$Y$中的有界集.$\exists c > 0, \st \forall x \in S, \norm{Tx} \leqslant c$,$\forall x \in X, x \neq 0, \frac{x}{\norm{x}} \in X$,故\[\frac{\norm{Tx}}{\norm{x}} = \norm{\frac{1}{\norm{x}} T x} = \norm{T(\frac{x}{\norm{x}})} \leqslant c\]

因此$\norm{Tx} \leqslant c \norm{x}, x \in X$.

$(2) \Rightarrow (3)$.设$\{x_n\} \subset X, x_n \to x$,由于$T$是线性的,因此$0 \leqslant \norm{Tx_n - Tx} = \norm{T(x_n - x)} \leqslant c \norm{x_n - x} \to 0.$因此$Tx_n \to Tx$.连续性得证.

$(3) \Rightarrow (1)$.若$T$不是有界的,则存在$X$中的有界集$A$,使得$T(A)$不是有界的.$\exists M > 0, \{x_n\} \subset A, \st \norm{x_n} \leqslant M, n \geqslant 1$,但是$\norm{Tx_n} > n$,\textbf{令}$x_n' = \frac{x_n}{\sqrt{n}}, n \geqslant 1$,则$x_n' \to 0$,则$x_n' \to 0$.由于$T$连续,应有$Tx_n' \to T(0) = 0$,但是
\begin{equation*}
	\norm{Tx_n'} = \norm{T(\frac{x_n}{\sqrt{n}})} = \frac{\norm{Tx_n}}{\sqrt{n}} \geqslant \sqrt{n} \to \infty.
\end{equation*}
矛盾,因此$T$有界.

\textbf{有界性通常可以由Holder不等式放缩得到.}

第二型Fredholm积分算子,\textbf{有限维}赋范空间上的算子是有界的.

无界的线性算子的例子:$C^{(1)}[0, 1]$上的微分算子$D: C^{(1)}[0, 1] \to C[0, 1], (Dx)(t) = x'(t)$.取$x_n(t) = t^n, n = 1, 2, \cdots$,则$\forall n, x_n \in C^{(1)}[0, 1], \norm{x_n} = 1, \norm{Dx_n} = \max_{0 \leqslant t \leqslant 1}{nt^{n - 1}} = n, n = 1, 2, \cdots$

\textbf{算子范数:}$X, Y$是赋范空间,$T:X \to Y$是有界线性算子,算子范数为:\[\norm{T} = \sup_{\norm{x} \leqslant 1}{\norm{Tx}}.\]

$f$是$X$上的有界线性泛函,范数为:\[\norm{f} = \sup_{\norm{x} \leqslant 1}{|f(x)|}.\]

算子范数的等价定义:$\norm{T} = \sup_{x \neq 0}{\frac{\norm{Tx}}{\norm{x}}} = \sup_{\norm{x} = 1}{\norm{Tx}}$.

证明:
\begin{align*}
	\sup_{\norm{x} \leqslant 1}{\norm{Tx}} & \leqslant \sup_{\norm{x} \leqslant 1, x \neq 0}{\frac{\norm{Tx}}{\norm{x}}} \leqslant \sup_{x \neq 0}{\frac{\norm{Tx}}{\norm{x}}} \\
				         				   & = \sup_{x \neq 0}{\norm{T(\frac{x}{\norm{x}})}} = \sup_{\norm{x} = 1}{\norm{Tx}} \leqslant \sup_{\norm{x} \leqslant 1}{\norm{Tx}}
\end{align*}

范数的性质:$\norm{Tx} \leqslant \norm{T} \norm{X}, x \in X$.同时,若$c > 0, \st \norm{Tx} \leqslant c \norm{X}$,则有$\norm{T} \leqslant c$.

符号函数:$x \in \mathbb{K}^1$,则
\begin{equation*}
	\sgn{x} = 
	\begin{cases}
		\frac{\bar{x}}{|x|}, x \neq 0 \\
		0, x = 0.
	\end{cases}
\end{equation*}

符号函数的性质:$x \sgn{x} = |x|, |\sgn{x}| \leqslant 1$.

\textbf{算子范数的计算:(两个方向放缩)}

设数列$a = (a_i) \in l^1$,$f(x) = \sum_{i = 1}^{\infty}{a_i x_i}, x \in l^{\infty}$为定义在$l^{\infty}$上的泛函.

显然$f$是\textbf{线性的}.$\forall x = (x_i) \in l^{\infty}$,有
\begin{equation*}
	|f(x)| \leqslant \sum_{i = 1}^{\infty}{|a_i x_i|} \leqslant |a_i| \sup_{i \geqslant 1}|x_i| = \nnorm{a}{1} \nnorm{x}{\infty}.
\end{equation*}

\textbf{这表明$f$有界,且$\norm{f} \leqslant \nnorm{a}{1}$},另一方面,令$x^{(0)} = (\sgn{a_1}, \sgn{a_2}, \cdots) \in l^{\infty}$,并且$\nnorm{x^{(0)}}{\infty} \leqslant 1$.由于
\begin{equation*}
	|f(x^{0})| = |\sum_{i = 1}^{\infty}{a_i x_i^{(0)}}| = |\sum_{i = 1}^{\infty}{a_i \sgn{a_i}}| = \sum_{i = 1}^{\infty}{|a_i|} = \nnorm{a}{1}.
\end{equation*}

因此$\norm{f} \geqslant \nnorm{a}{1}$,综上所述$\norm{f} = \nnorm{a}{1}$.

若$f$视作$c_0$上的泛函,正向不等式与上述证明一致,反向不等式:$\forall \varepsilon > 0, \exists k \in \mathbb{N}, \st \sum_{i = 1}^{k}{a_i} > \nnorm{a}{1} - \varepsilon$,取$x^{(0)} = (\sgn{a_1}, \sgn{a_2}, \cdots, \sgn{a_k}, 0, \cdots) \in c_0$,并且$\nnorm{x^{(0)}}{\infty} \leqslant 1$.由于
\begin{equation*}
	|f(x^{0})| = |\sum_{i = 1}^{\infty}{a_i x_i^{(0)}}| = |\sum_{i = 1}^{\infty}{a_i \sgn{a_i}}| = \sum_{i = 1}^{\infty}{|a_i|} = \nnorm{a}{1} - \varepsilon.
\end{equation*}

由$\varepsilon > 0$的任意性,$\norm{f} \geqslant \nnorm{a}{1}$,综上所述$\norm{f} = \nnorm{a}{1}$.

有界线性算子的空间:用$B(X, Y)$表示从$X$到$Y$的有界线性算子的全体.$\forall A, B \in B(X, Y), \alpha \in \mathbb{K}$,定义:
\begin{equation*}
	(A + B)x = Ax + Bx, (\alpha A)x = \alpha A x, x \in X.
\end{equation*}

又$\norm{(A + B)x} = \norm{Ax + Bx} \leqslant \norm{Ax} + \norm{Bx} \leqslant (\norm{A} + \norm{B})\norm{X}$,因此$A + B$是有界的,且有$\norm{A + B} \leqslant \norm{A} + \norm{B}$.

验证该范数是良定义的:
\begin{itemize}[itemindent=2em]
	\item $\norm \leqslant 0$,并且$A = 0 \Rightarrow \norm{A} = 0$,若$\norm{A} = 0$,则$Ax = 0, x \in X$(算子范数的不等式),则$A = 0$.
	\item $\forall \alpha \in \mathbb{K}$,$\norm{\alpha A} = \sup_{\norm{x} \leqslant 1}{\norm{(\alpha A) x}} = \sup_{\norm{x} \leqslant 1}{|\alpha| \norm{Ax}} = \alpha \norm{A}$.
	\item $\norm{A + B} \leqslant \norm{A} + \norm{B}$由之前的式子可得.
\end{itemize}

若$Y$是Banach空间,则$B(X, Y)$是Banach空间.

证明:设$\{T_n\}$是$B(X, Y)$中的Cauchy序列,则$\forall \varepsilon > 0, \exists N, \st m, n > N, \norm{T_m - T_n} < \varepsilon$.于是$\forall x \in X, m, n > N$,\[\norm{T_m x - T_n x} = \norm{(T_m - T_n)x} \leqslant \norm{T_m - T_n} \norm{x} < \varepsilon \norm{x}.\]

这表明$\{T x_n\}$是$Y$中的Cauchy序列,由于$Y$是完备的,$\{T_n x\}$收敛,$\forall x \in X$,令$Tx = \lim_{n \to \infty}{T_n x}$,则$T$是$X$到$Y$的线性算子,固定$n > N$,令$m \to \infty$,得到$\norm{(T - T_n)x} = \norm{Tx - T_n x} \leqslant \varepsilon \norm{x}, x \in X$.

上式表明$T - T_n \in B(X, Y)$.由于$B(X, Y)$是线性空间,$T = (T - T_n) + T_n \in B(X, Y)$,且$\norm{T - T_n} \leqslant \varepsilon.$故$B(X, Y)$是完备的.

\end{document}
