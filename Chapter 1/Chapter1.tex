\documentclass[bwprint, withoutpreface]{cumcmthesis}

\usepackage{extarrows}

\title{实变第一章总结}

\begin{document}
\maketitle
\section{集合与集合的运算}
\indent 集合的表示:列举法,描述法.

子集,真子集.

\textbf{幂集:} $\mathcal{P}(X) = \{A:A \subset X\}$, 若$|X|=n$,则$|\mathcal{P}(X)|=2^n$.

集合的有限交,并,差,余.

相对差集:$A \Delta B = (A - B) \cup (B - A)$,若$A \Delta B = \emptyset$,$A = B$.

集族,集列:若对$\forall \alpha \in I$都对应一个集$A_{\alpha}$,则称$\{A_n\}_{\alpha \in I}$为集族.若$I = \mathbb{N}$,则称$\{A_n\}_{\alpha \in \mathbb{N}}$为集列.

集合的可列交:
\begin{equation*}
	\bigcup_{\alpha \in I}{A_\alpha} = \{\exists \alpha \in I, s.t. \quad x \in A_{\alpha}\}
\end{equation*}.

集合的可列并:
\begin{equation*}
	\bigcap_{\alpha \in I}{A_\alpha} = \{\forall \alpha \in I, x \in A_{\alpha}\}
\end{equation*}.

可列交与可列并的性质:交换律,结合律,分配律.

\textbf{De Morgan 公式:}
\begin{align*}
	{(\bigcup_{\alpha \in I}{A_{\alpha}})}^C = \bigcap_{\alpha \in I}{{A_{\alpha}}^C} \\ 
	{(\bigcap_{\alpha \in I}{A_{\alpha}})}^C = \bigcup_{\alpha \in I}{{A_{\alpha}}^C}
\end{align*}

\textbf{集合的表示:}根据所给条件由内向外进行表示.一般等价转化为:$\forall: \bigcap$,$\exists: \bigcup$.通过对\textbf{指标依赖的分析},依次将极限定义转化为集合符号的定义。

直积:有序$n$元组全体构成的集:$\{ (x_1, x_2, \cdots, x_n): x_1 \in A_1, x_2 \in A_2, \cdots, x_n \in A_n \}$

集列的极限:

上极限:
\begin{equation*}
	\varlimsup_{n \to \infty} = \{x: x\mbox{属于}A_n\mbox{中的无限多个}\}
\end{equation*}.

下极限:
\begin{equation*}
	\varliminf_{n \to \infty} = \{x: x\mbox{至多不属于}A_n\mbox{中的有限多个}\}
\end{equation*}.

\begin{equation*}
	\varlimsup_{n \to \infty} \subset \varliminf_{n \to \infty}
\end{equation*}

若
\begin{equation*}
	\varlimsup_{n \to \infty} = \varliminf_{n \to \infty}
\end{equation*}

则称$\{A_n\}$存在极限,记
\begin{equation*}
	A \xlongequal{def} \varlimsup_{n \to \infty} = \varliminf_{n \to \infty}
\end{equation*}

\textbf{集列极限的表示:上极限先变大再变小,下极限先变小再变大.}
\begin{align*}
	\varlimsup_{n \to \infty} = \bigcap_{n = 1}^{\infty}{\bigcup_{k = n}^{\infty}{A_k}} \\
	\varliminf_{n \to \infty} = \bigcup_{n = 1}^{\infty}{\bigcap_{k = n}^{\infty}{A_k}}
\end{align*}

\textbf{单调集列必存在极限,且:}
\begin{align*}
	\lim{A_n} = \bigcup_{n = 1}^{\infty}{A_n} \quad A_n\mbox{单调递减} \\
	\lim{A_n} = \bigcap_{n = 1}^{\infty}{A_n} \quad A_n\mbox{单调递增}
\end{align*}

\section{映射,可列集,基数}
\indent 映射,值域,定义域,像,原像,函数,单射,满射,双射,逆映射,反函数,复合映射,延拓,限制.

\textbf{特征函数(示性函数):}

$A \subset X$,
\begin{equation*}
	\chi_A(x) = 
	\begin{cases}
		1, \quad x \in A \\
		0, \quad x \not \in A		
	\end{cases}	
\end{equation*}

特征函数的性质.

\textbf{特征函数能够表示分段函数,可以看作对函数的一种“粘连”:}
\begin{equation*}
	f(x) = \sum_{i = 1}^{n}{f_i(x)\chi_{A_i}(x)}
\end{equation*}

其中$\{A_n\}$为不相交的子集,并且$X$为它们的并集.$f_i(x)$为定义在$\{A_i\}$上的函数。

\textbf{集合的对等关系:}$A$,$B$是两个非空集合,若$\exists \phi:A \to B, \phi\mbox{为双射}$,则$A$,$B$对等,记为$A \sim B$.且$\emptyset \sim \emptyset$.

集合的对等关系满足自反性,对称性与传递性.

\section{集类}
\section{$\mathbb{R}^n$中的点集}

\end{document}